\documentclass[a4paper,12pt]{article}
\usepackage[utf8]{inputenc}
\usepackage[english,russian]{babel}
\usepackage[T2A]{fontenc}
\usepackage{amssymb}
\usepackage{amsthm}


\usepackage[
  a4paper, mag=1000, includefoot,
  left=1.1cm, right=1.1cm, top=1.2cm, bottom=1.2cm, headsep=0.8cm, footskip=0.8cm
]{geometry}

\usepackage{amsmath}
\usepackage{amssymb}
\usepackage{times}
\usepackage{mathptmx}
\usepackage{graphicx}

\IfFileExists{pscyr.sty}
{
  \usepackage{pscyr}
  \def\rmdefault{ftm}
  \def\sfdefault{ftx}
  \def\ttdefault{fer}
  \DeclareMathAlphabet{\mathbf}{OT1}{ftm}{bx}{it} % bx/it or bx/m
}


\mathsurround=0.1em
\clubpenalty=1000%
\widowpenalty=1000%
\brokenpenalty=2000%
\frenchspacing%
\tolerance=2500%
\hbadness=1500%
\vbadness=1500%
\doublehyphendemerits=50000%
\finalhyphendemerits=25000%
\adjdemerits=50000%
\newcommand {\pp} {\partial}
\newcommand {\ppr} [4] {\frac{\pp {#1} ^{#2}}{\pp {#3} ^{#4}}}
\newcommand {\cd} {\cdot}
\newcommand {\SL} {\implies}
\newcommand {\HD} {\hdots}
\newcommand {\FI} {\varphi}
\newcommand {\eps} {\epsilon}
\newcommand {\we} {\wedge}
\newcommand {\st} {\longleftarrow}
\newcommand {\titt} {\Longleftrightarrow}
\newcommand {\trr} {\blacktriangleright}
\newcommand {\trl} {\blacktriangleleft}
\newcommand {\iii} {\int\limits}
\newcommand {\ido} {\iint\limits} 
\newcommand {\AL} {\alpha}
\newcommand {\AS} {\alpha'}
\newcommand {\ppp}[2] {\ppr{x}{#1}{x}{#2}}
\newcommand {\LS} {\sum\limits}
\newcommand {\bb} {\mathbb}
\newcommand {\ta} {\theta}
\newcommand {\TA} {\Theta}
\newcommand {\pt} {\ppr{}{}{\ta}{}}
\newcommand {\mm} {\mathcal}

\theoremstyle{plain}
\newtheorem{predl}{Предложение}
\newtheorem{lem}{Лемма}
\newtheorem{teo}{Теорема}
\newtheorem{opr}{Определение}


\begin{document}

\author{сфс}
\title{лекции по теории случайных процессов 2023}
\date{\today}
\maketitle
\large
\newpage
\section{\centering{Лекция 1}}
Процесс Бернулли. Случайные блуждания.\\
Понятие случайного процесса.\\\\
\subsection{\centering{Путь развития тервера}}
\MakeUppercase{\romannumeral 1} подсчет вероятностей некоторых событий\\
\MakeUppercase{\romannumeral 2}  исследование случайных величин\\
\MakeUppercase{\romannumeral 3}  случайные процессы (стали изучать динамику случайных величин в пространстве и времени)\\\\
$(V, \mathcal {A})$ -- измеримое пространство, где $V$ -- множество, $\mathcal {A}$ -- $\sigma$-алгебра\\
/*свойства $\sigma$-алгебры
\begin{enumerate}
\item $A\in \mathcal {A}\SL \overline{A}\in\mathcal {A}$
\item $\Omega\in \mathcal {A}$; $\varnothing \in \mathcal {A}$
\item $A_1, A_2, ..., \in \mathcal {A}\SL \bigcup\limits_i A_i\in \mathcal {A}$
\end{enumerate}
*/\\
$(\Omega, \mathcal{F}, P)$ -- вероятностное пространство, где $\Omega$ -- множество, $\mathcal{F}$ -- $\sigma$-алгебра, $P$ -- вероятностная мера. \\/*свойства вероятностной меры\\$A_1, A_2,... \in \mathcal{F}, A_i\cap A_j = \varnothing \SL P(\bigcup\limits_{n=1}^\infty A_n) = \LS_{n=1}^\infty P(A_n)$\\ $P(\Omega) = 1$*/\\\\
Рассмотрим измеримые пространства $(V,\mathcal{A})$ и $(S, \mathcal{B})$\\
$f:V\to S$, говорят, что $f\in\mathcal{A}|\mathcal{B}$ ($\mm{A} - \mm{B}$ - измеримая), если $\forall B\in\mm{B}: f^{-1} (B) = \left\{\ x \in V: f(x)\in B\right\}\in \mm{A}$
\begin{opr}
Пусть $(S_t, \mm{B}_t)_{t\in T}$ -- семейство измеримых пространств, где $T$ - любое множество. Случайный процесс $X = \{X(t), t\in T\}$ - семейство отображений $X(t): \Omega \to S_t$: $X(t)\in \mm{F}|\mm{B}_t \quad \forall t\in T$\\
$X(t)$ называют случайным элементом (случайная величина, но со значениями в $S_t$)
\end{opr}
Обычно $S_t = S; \mm{B} = \mm{B}$\\
Важно:\\
1) Все $X(t)$ определены на одном и том же $\Omega$\\
2) $X(t)\in\mm{F}|\mm{B}_t \forall t$\\
Другими словами - случайный процесс -- это семейство случайных величин. (если $S = \bb{R}, \mm{B} = \mm{B}(\bb{R})$)

Если $T \subset R$, то точки $t$ интерпретируются как время.\\
Обычно в качестве $T$ берут $\bb{Z}, \bb{Z}_+, \bb{N}, \bb{Z}\cap[a,b], h\bb{Z}$ в дискретном случае или $\bb{R}, \bb{R}_+, [a,b]$ в непрерывном случае.\\
Введём пространство $S_T = \prod\limits_{t\in T}S_t$\\\\


\begin{opr}
Oпределим элементарный цилиндр $C(t, \mm{B}_t) = \{x\in S_t: x(t)\in \mm{B}_t\}$
\end{opr}
$\mm{B}_T = \sigma\{$элементарные цилиндры$\}$ -- цилиндрическая сигма алгебра.\\
$C_{t_1,...,t_n}(B_1,...,B_n) = \bigcap\limits_{k=1}^{n} C(t_k, B_k)$

\begin{opr}
Пусть $\omega\in \Omega$ фиксировано. Функция $X(t,\omega),t\in T$ называется траекторией (/реализацией/выборочной функцией) случайного процесса.
\end{opr}
Упражнение: доказать, что $X$ - случайный процесс, ассоциированный с семейством $S_t, \mm{B}_t)_{t\in T} \titt X\in \mm{F}|\mm{B}_T$\\\\

\begin{opr}
Рассмотрим $\forall n, \forall \{t_1,t_2,...,t_n\}\subset T$. Рассмотрим $(X_{t_1}, ..., X_{t_n})_\omega$ на $(S_{t_1}\times ... \times S_{t_n}, \mm{B}_1\times ... \times \mm{B}_n)$. Такие $P_{X_{t_1,...,t_n}}$ называют конечномерными распределениями процесса.
\end{opr}

\begin{opr}
Семейство $\{X_t, t\in T\}$ состоит из независимых случайных элементов, если $\forall n\in N, \forall \{t_1,...,t_n\}\subset T, \forall B_k\subset \mm{B}_{t_k}:$
$$P(X_{t_1} \in B_1, ..., X_{t_n}\in B_n) = \prod\limits_{k=1}^n P(X_{t_k}\in B_k)$$
\end{opr}

\begin{teo}[Ломницкого-Улама]
Пусть $(S_t, \mm{B}_t)_{t\in T}$ - любое семейство измеримых пространств. Пусть $Q_t$ - произвольная вероятностная мера на $(S_t, \mm{B}_t)$. Тогда на некотором $(\Omega, \mm{F}, P)$ существует семейство независимых случайных элементов $X_t:\Omega\to S_t, X_t\in \mm{F}|\mm{B}_t, t\in T$ такое, что $P_{X_t} = Q_t$\\
Иначе говоря, на некотором вероятностном пространстве можно построить семейство независимых элементов с заданными распределениями.
\end{teo}

\subsection{\centering{Процесс Бернулли}}
\begin{opr}
Последовательность из случайных величин $X_1, X_2, ...$ называют процессом Бернулли, если $P(X_n = 1) = p, P(X_n = 0) = 1-p, p\in(0;1)$
\end{opr}
Рассмотрим последовательность бросков, например 010010111\\
$n=9; L_n=3$ - серия из 1 максимальной длины\\
Результат Batemon 1948:\\
$P(L_n\ge m) = \LS_{j=1}^\infty (-1)^{j+1} (p+\frac{n-mj+1}{j}q) C_{n-mj}^{j-1} p^{mj}q^{j-1}$\\\\


\begin{teo}[Эрдеш, Ревес]
С вероятностью 1: $\lim\limits_{n\to\infty}\frac{L_n}{log_{\frac{1}{p}}(n)}= 1$
\end{teo}


\subsection{\centering{Случайные блуждания}}
\begin{opr}
Пусть $X_1, X_2, ...$ н.о.р.с.вектора со значениями в $\mathbb{R}^d$. Процесс вида $S_n = x + X_1 + X_2 + ... + X_n$ называют случайным блужданием.
\end{opr}
Если $X_k\in \mathbb{Z}^d$ то случайное блуждание на сетке.\\
далее будем рассматривать $x=0$

\begin{opr}
Случайное блуждание на $\mathbb{Z}^d$ называют возвратным, если с вероятностью 1 процесс $S_n$ возвращается в начальную точку бесконечное количество раз.
\end{opr}

\begin{opr}
Блуждание невозвратно, если с вероятностью 1 лишь конечное число возвращений.
\end{opr}

Введём $N:=\LS_{n=0}^\infty I\{S_n=0\} (\le\infty)$\\
Пусть $\tau := inf\{n\in \mathbb{N}:S_n = 0\}, \tau := \infty$, если все $S_n \ne 0$

\begin{lem}
$$\forall n \in N: P(N=n) = P (\tau = \infty)P(\tau<\infty)^{n-1}$$\\(полагаем $0^0 = 1$)
\end{lem}









\input{lec2.tex}
\input{lec3.tex}
\input{lec4.tex}
\input{lec5.tex}
\input{lec6.tex}
\input{lec7.tex}
\input{lec8.tex}
\input{lec9.tex}
\input{lec10.tex}
\input{lec11.tex}
\input{lec12.tex}
\input{lec13.tex}
\input{lec14.tex}


\end{document}

